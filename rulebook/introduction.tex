% (c) 2020 Stefan Antonowicz
% Based off of tex found at https://github.com/ludus-leonis/nipajin
% This file is released under Creative Commons Attribution-NonCommercial-ShareAlike 4.0 International License.
% Please do not apply other licenses one-way.



\renewcommand{\yggIntroductionText}{%

\hrulefill


\center\Huge\fffancy{The Totality of Ygg}\fftext\normalsize


\mybold{An OSR rules variant}

Written and designed by \mybold{Stefan Antonowicz}

Cover design by \mybold{vil}

Spriggan and Nightchildren artwork by \mybold{Anastasia Ordonez}

Playtesters were Anastasia Ordonez, Eric Coumbe, Javier Peña, Rob Rea, Bobby Plank, Lou Coehlo, Cardell Kerr, Shane Burke, Rob Zasso, and Dave Herling.  You guys - I couldn't have done it without you.  Thanks for your input and patience.

I used \LaTeX for the layout.  The source code is available at \href{https://github.com/gadhra/totalityofygg}{https://github.com/gadhra/totalityofygg}

All other artworks (except the cover) are in the public domain.  Original works by Ulisse Aldrovandi, Gustave Dore, Harry Clarke, Henry Justice Ford, John Batten, Piranesi Carseri, Thomas Hood, X.B. Saintine, Guy Boothby, Martin Farquhar Tupper, Gaston Vuillier, Luigi Robecchi-Bricchetti and others.


I'm going to paraphrase Emmy Allen here, because intellectual property law is hard and it's giving me a headache and I've got games I want to play.  The rules for the Totality of Ygg are built on the shoulders of giants; they will always and forever be free (if I ever Kickstart a physical copy or what-have-you, I promise that any money raised will go towards artists and printing/shipping costs and that's it).  I've included the OGL at the end of the book to respect those who published \mybold{their} works under it, and linked to folks I've "borrowed" from (in the Pablo Picasso sense) in \mylink{Appendix N}{appendix-n} and in the text - but otherwise feel free to steal, copy, and reproduce as you wish.  Just be cool (please) and put what you make back into the public domain. 


\hrulefill

    So, what can you expect?

    \mybold{"Dungeon Master" friendly}.   I've been the group DM for forever.  I've spent many, many (so many) hours prepping, statting, rolling dice, and trying to create an adventure 2 hours before the game is supposed to start.  These rules are my answer to problems and issues I've seen pop up time and again in games I've run.  Players roll most of the dice (which eases narration for the DM); balanced monster stats can be generated quickly and easily on the fly; and there are multiple resource levers you can pull (instead of just hit points, XP, and money) to give your players a good challenge with plenty of middle ground.

    \mybold{Medium "Crunch"}.  The ToY rules lie in the middle ground between high and low crunch.  While I've got great memories of someone at the table poring over a rulebook and having a sudden "Eureka!" moment, I've also got less happy memories of certain unnamed games losing themselves in the complex rules of their cyberpunk future (random example).  There are plenty of charts and plenty of dice rolling, but I've tried to keep the door open to the "rulings not rules" approach that I think is the most fun

    \mybold{"Old School" feel}. OSR has been an absolute embarrassment of riches for the hobby; the involvement of artists in the community has mixed the joy of those old Red Box games with an inventiveness and creativity that has had me jump out of my seat more than once.  OSR is how I remember playing the 80s (or, at least, how I \mybold{wish} my games had been in the 80s).

    \mybold{Free}. The rules for the Totality of Ygg will always and forever be free.


    I hope you enjoy these rules as much as I did writing them.  May your games be fun, your memories endless, and your friendships eternal!

    \vspace*{\fill}
    \justifying

} %end
