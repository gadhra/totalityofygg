% (c) 2020 Stefan Antonowicz
% Based off of tex found at https://github.com/ludus-leonis/nipajin
% This file is released under Creative Commons Attribution-NonCommercial-ShareAlike 4.0 International License.
% Please do not apply other licenses one-way.

\renewcommand{\yggLeech}{%
  \mychapter{Leeches}{leeches}
}

\renewcommand{\yggLeechText}{%

  \mysection{Knowledge }{leech-knowledge}

  \mytable{l X}{
    \thead{Level} & \thead{Knowledge} \\
  }{
    1 & d8 \STATIC \\
    2-3 &  d10 \STATIC \\
    4-5 &  d12 \STATIC \\
    6-7 &  d16 \STATIC \\
    8 &  d20 \STATIC \\
    9 &  d24 \STATIC \\
  }

  The Crux of Knowledge is represented by a the \STATIC Knowledge Die.  Sometimes you may be asked to make a Knowledge Test.  To do so, \RO using your Knowledge Die plus \INT and any other modifiers.
  \example {
    \mybold{Knowledge Test}

    \RO: \INT plus \mybold{Knowledge Die} plus \mybold{Modifiers}

  }

  Any time you fail a Knowledge Test, you begin to accumulate negative modifiers.  The modifiers reset once you take a Bivouac or longer rest.

  \mytable{l X} {
    \thead{Failure} & \thead{Modifier} \\
  } {
      1st  & -1 \\
      2nd  & -1 \\
      3rd  & -2 \\
      4th  & -3 \\
      5th  & -5 \\
      6th  & -8 \\
      7th  & -13 \\
      8th  & etc. \\
}

\cbreak

\mysection{Research}{leech-research}

Research is used to perform \mylink{Chymistry}{leech-chymistry}, and can be used to help Sorcerers practice \mylink{Inscription}{sorcerer-inscription}.  Your Research is equal to your maximum \INT (so if you have a d12 \INT, you have 12 Research). You may only perform Research in a Civilization with a library 

If you take a Sabbatical, the number of Research Die you have is tripled (so 12 Research becomes 36).

Use your Research to "build" one or more \KNACK called Research Die. You can "build" as many Research Die as you choose up to your Research total. For example, if you have 12 Research, you could create:

\mybullet {
  \item 2 6-in-6 Research Die
  \item 3 4-in-6 Research Die
  \item 6 2-in-6 Research Die
  \item etc.
}

Remember that a Research Die is a \KNACK.  If your Research Die is a 2-in-6, you get a success if you roll a 1 or a 2 on a d6; if the die is 6-in-6, you roll 2d6 and only fail on a 2 (snake eyes).  You can only roll each Research Die once per Sabbatical.

Research Dice can be traded to \mylink{Sorcerers}{sorcerers} as well as to other Leeches (knowledge is shared, after all). You can give as many of your Research Dice to different people as you wish, but you can only accept a number of Research Dice up to your \LVL-1.

\newpage

\mysection{Leechcraft}{leech-leechcraft}

The skills of healing and mending.  To perform Leechcraft, you must make a \mylink{Knowledge Test}{leech-knowledge}

  \example {
    \mybold{Knowledge Test}

    \RO: \INT plus \mybold{Knowledge Die} plus \mybold{Modifiers}

  }



  You can perform Leechcraft as many times as you'd like, but remember that failed Knowledge Tests add progressively higher negative modifiers.

  Note that Leechcraft will only fix temporary effects (usually an effect with a \mylink{Markovian}{duration-markovian} duration) unless the \mylink{Purge}{keyword-purge} keyword is indicated.

  Finally, Leechcraft with a length of Minutes can be performed during a Breather, and Hours can be performed during a Bivouac.  If taking a Bivouac, you must roll \mybold{before} you reset the negative modifiers on your Knowledge.  You can try again after you reset your negative modifiers, but this would require another Bivouac.  

 
  \LEECHCRAFT[
    Name=Bonesetting,
    Link=leechcraft-bonesetting,
    Mod=+3,
    Length=Hours,
    Keywords=Purge
  ]
  Purge a single non-serious Physical Wound

  \LEECHCRAFT[
    Name=Delay Infection,
    Link=leechcraft-delay-infection,
    Mod=+3,
    Length=2 Maneuvers,
    Keywords=None 
  ]
  
  Delay the onset of a single Disease (including contagion) for Hours.  Curing the disease requires \mylink{Medicinals}{leech-medicinals}

  \cbreak

  \LEECHCRAFT[
    Name=Hair of the Dog,
    Link=leechcraft-hair-of-the-dog,
    Mod=+9,
    Length=Minutes,
    Keywords=Purge 
  ]
  Purge the effect of a Hang Over on a single person

  \LEECHCRAFT[
    Name=Laudanum,
    Link=leechcraft-laudanum,
    Mod=+0,
    Length=see below,
    Keywords=Purge 
  ]
  Purge a single non-serious Mental Wound (Minutes) OR remove a Disgusted, Shaken, or Sickened effect (1 Maneuver) on a single patient

  \LEECHCRAFT[
    Name=Mend,
    Link=leechcraft-mend,
    Mod=+9,
    Length=1 Maneuver,
    Keywords=None 
  ]
  Can only be applied to a patient on Death's Door. Heal 1 Flesh

  \LEECHCRAFT[
    Name=Purge Toxin,
    Link=leechcraft-purge-toxin,
    Mod=+3,
    Length=1 Maneuver,
    Keywords=Purge 
  ]
  Immediately purge a toxin from the body of a single patient

  \LEECHCRAFT[
    Name=Restore Senses,
    Link=leechcraft-restore-senses,
    Mod=+6,
    Length=2 Maneuvers,
    Keywords=None 
  ]
  Remove a Blindness or Deafness effect on a single patient

  \LEECHCRAFT[
    Name=Sew Wounds,
    Link=leechcraft-sew-wounds,
    Mod=+9,
    Length=Minutes,
    Keywords=None 
  ]
  The patient rolls 1 \FLESH and heals that much Flesh. You can only do this once per patient during a Breather or Bivouac (before you erase your Knowledge failures).

  \LEECHCRAFT[
    Name=Smelling Salts,
    Link=leechcraft-smelling-salts,
    Mod=+6,
    Length=2 Maneuvers,
    Keywords=Purge 
  ]
  Purge a Knocked Out effect on a single patient


  \LEECHCRAFT[
    Name=Staunch,
    Link=leechcraft-staunch,
    Mod=+6,
    Length=1 Maneuver,
    Keywords=Purge 
  ]
  Purge a Bleed effect on a single patient

  \cbreak

  \LEECHCRAFT[
    Name=Trepanation,
    Link=leechcraft-trepanation,
    Mod=+0,
    Length=Hours,
    Keywords=Purge 
  ]
  Purge a Woozy effect on a single patient

  \LEECHCRAFT[
    Name=Virtigo,
    Link=leechcraft-virtigo,
    Mod=+6,
    Length=Hours,
    Keywords=Purge 
  ]
  Purge a Befuddled or Concussed effect on a single patient

  \newpage

\end{multicols}

\mysection{Medicinals}{leech-medicinals}



\myital{These are pretty much verbatim from Logan Knight's amazing \href{https://www.lastgaspgrimoire.com/does-this-look-infected}{Does This Look Infected?}}

\hrulefill



Medicinals are practiced during Sabbatical to purge Disease, Addiction, and Madness as well as repair the effects of a serious Physical wound.  Leeches are able to purge these afflictions using a Research Die.

\mynumlist {
  \item State how many Research Die you wish to roll.  These will be rolled regardless of what happens in step \#2
  \item Roll a d16 and consult the chart below.  If you do not have enough coins, the Medicinal immediately fails and victim suffers the \mybold{Misdiagnosis}
  \item \RS your Research Dice.  If the number of failures exceed the number of successes, you have a Misdiagnosis: the Wound, Disease, Addiction, or Madness is not cured and you lose the money invested. 
}

You can try for as long as you have Research Die.

If you don't have a Leech in your party, you can find one (or more) in a Sanitarium (see Core Rules). The Research Die used by a leech or wisewoman you find are the same as the odds of finding a Sanitarium (i.e. 2-in-6 for Medium and 6-in-6 for Large), and the cost in coins is the same.

The \COST depends on the Civilization - Iron (Small), Silver (Medium), or Gold (Large)


\mynumlist {
  \item \COST 10.  The patient must eat the worm-infested testicles of a rabid baboon. They populate quickly and enter the bloodstream, consuming insantities and diseases. Before the worms turn to feast upon healthy tissue, their veins will be opened so that they fall from them onto an open flame. 

\myital{Misdiagnosis:  Not all of the worms are removed, which makes things rather uncomfortable for the patient}

  \hrulefill

  \item \COST 10.  A clyster of hot eel blood and the crushed eggs of a Horned Soldier Crab. 

\myital{Misdiagnosis:  Not all of the eggs were crushed, after a week they begin to hatch and leave their body}

\hrulefill



  \item \COST 15. An incision must be made along the abdomen  with a golden blade. After it has been allowed to weep the wound will be smeared with saccharine nectar squeezed from the abdomen of Larder Ants, then sutured with the final dying bite of their heads. 

\myital{Misdiagnosis:  No matter how hard they try flies keep getting into the wound. That's how you get maggots.} 

\hrulefill



  \item \COST 15. Swallow a ward-inscribed lead ball, still glowing from the fire. 

\myital{Misdiagnosis:  The ball sears their esophagus on the way down; say goodbye to solid food for at least a week.}

\hrulefill



  \item \COST 20. Bloodletting of one or more limbs. Once venesected, the blood must be allowed to flow into heated iron bowls to be boiled away, banishing the maladies in acrid smoke. 

\myital{Misdiagnosis:  That was a little too much blood. They lose 1 Flesh permanently}

\hrulefill



  \item \COST 20. Sin-Eater. It is an uncleanliness of the spirit that has caused the blight to manifest. A layer of skin must be flayed from the thigh and offered to the Leech, who will sear it over a flame for consumption along with a cup of blood from the afflicted. 

\myital{Misdiagnosis:  you are overcome with revulsion and throw the victim out onto the street without further explanation}

\hrulefill



  \item \COST 25. The purifying qualities of quicksilver will remove the taint from them. Thin, steady streams are to be poured into the eyes, windows of the soul. 

\myital{Misdiagnosis:   They are inflicted with a random Insanity}

\hrulefill



  \item \COST 30. A limb is covered with grubs sourced from the rotting undergrowth of the Grimm Wood, who consume a few dozen mouthfulls of flesh, transfering the malady to themselves and falling dead.The hardened bodies should then be ground to a fine powder and insufflated to prevent the infection from returning. 

\myital{Misdiagnosis:  When you grind the bodies of the grubs you finds them rock hard. In frustration you take to them with a hammer and they shatter into red crystalline shards.}

\hrulefill



  \item \COST 30. Upon a stone slab, lathered with the blood honey of the spore-crazed bees from deep within the Veins of the Earth, flaming torches will be passed over your body until the moisture has been sapped from the honey. Once it has cooled and crystallised from your skin the malady will be gone. 

\myital{Misdiagnosis:  For the next week the patient awakens each night at the edge of their roof, toes overhanging and ready to fall.}

\hrulefill



  \item \COST 35. The malady is inside of your stomach. You must swallow a live Hagora Fish caught in the foetid, sluggish streams of the Weald. As it digests your diseased stomach lining it will defecate the remedy into your bloodstream.  

\myital{Misdiagnosis:  Before it is digested the fish continues to devour your stomach lining once the diseased flesh is gone. Take a -1 to their Max Flesh.}

\hrulefill



  \item \COST 40. The thumb-sized yellow mould leeches from the stagnant pools of the Quenchless Mouth of Many can drain the corruption from your blood, but only if kept sufficiently warm and moist while they feed over two days. You may attempt to take them in your mouth, but it is recommended as a suppository. 

\myital{Misdiagnosis:  Swollen with blood, the leeches block whichever passage you chose to deposit them in. They will need to be burnt out.}

\hrulefill



  \item \COST 50. An excess of phlegm is the cause of your infection and discomfort. Four golden tubes will be tapped through your chest and into your lungs before fastening a bellows-mask over your face, forcefully pushing air down your throat in order to discharge the malady. 

\myital{Misdiagnosis:  The initial trickle of phlegm stops with a muted thump through the golden tubes. You heave on the bellows-mask and a writhing mass of black maggots discharge into the bowl.}

\hrulefill



  \item \COST 50. Trepanation of the skull should relieve the affliction. 

\myital{Misdiagnosis:  The patient should probably wear a helmet for awhile}

\hrulefill



  \item \COST 50. The malady persists because it dwells in the heart, pumping its filth throughout your body. Your lips will be painted with the venom of the Blue Lesionaire Spider, and when your vitals slow your heart must be removed and pierced with heated copper needles, draining it of trapped black bile.  

\myital{Misdiagnosis:  The venom eats away part of the patient's lips, making speaking difficult}

\hrulefill



  \item \COST 75. It is an overabundance and corruption of yellow bile that causes your fever and rot. An incision must be made below the ribs to allow the tainted bile to be expressed from the gallbladder. 

\myital{Misdiagnosis:  You discover a tumor full of teeth and auburn hair nestled below the patients ribs. A Sorcerer can use this as a spell component, but it's unsure what the effect might be on the patient ...}

\hrulefill



  \item \COST 100. You have the evil eye. It offends with its infection and must be plucked from you. 

\myital{Misdiagnosis:  Whoops.  Apologies for the error, would you like a false replacement or an experimental transplant? No charge!}

\hrulefill


}

\newpage

\begin{multicols}{2}

\newpage

\mysection{Chymistry}{leech-chymistry}

The science of the transformation of matter, Chymistry can only be practiced in Civilization during a Sabbatical.  There are 4 basic kinds of chymicals:

\mylist {
  \item \mybold{Tonics}  are mixtures of booze, narcotics, and other things you probably don't want to know about. They must be drunk
  \item \mybold{Powders} can be drunk (in wine), inhaled, or smoked.  If you blow a powder in some Monster’s face, the Monster has to be able to breathe (so they don't work on undead, for example)
  \item \mybold{Sera} need to be injected via a syringe.  If the recipient is a person, they have to have a vein of some kind.  If the recipient is an object, it has to be something a needle could pierce (like an apple)
  \item \mybold{Ungeants} include oils, salves, lubricants, and pastes - basically anything you rub on things (heh heh).  They're usually super viscous and anyone will know the moment they try to drink it.  You can coat a weapon with an unguent  and "apply" it to a Monster by injuring them with the weapon (1 point of damage or more to Flesh).  If your attack misses the unguent doesn't rub off, but if you hit armor / scales / etc. without piercing Flesh, it does.  Unguents can be rubbed on someone who is who is unconscious or petrified.  Can’t be eaten -  tastes shitty and the person will spit it out.
}


\mylink{Research}{leech-research} \KNACK for each potion you wish to create.  If you fail you are unable to create the potion and the money spent on materials is lost. 

Potions are divided into 3 groups - Iron, Silver, and Gold.  \mybold{Each potion, toxin, or acid costs 500 coins in materials}, depending on the group they are in i.e. 500\FE, 500\AG, or 500\AU. If you succeed, you generate d4 \UD of the chymical.

\mysubsection{Such Mortal Drugs I Have (Toxins)}{leech-chymistry-toxins}

You can create a Toxin in the form of a Tonic, Unguent, Sera, or Powder. The efficacy of the potion depends on the coins spent:

  \mytable{X X r} {
    \thead{\COST} & \thead{Damage \& Duration} & \thead{Saves} \\
  } {
    500\FE  & d4 & 1 \\
    500\AG  & d8 & 2 \\
    500\AU & d12 & 3 \\
  }

  The die type is the amount of damage over a number of Minutes, rolled separately. Thus, a Silver Toxin would deal d8 damage every Minute for d8 Minutes (the Arbiter is encouraged to roll the duration in secret). Damage hits Grit first (as it wears away your will to live), then Flesh.  You cannot heal Grit while under the effects of a Toxin.  The victim of these Toxins \mybold{always} gets a Save (though see below). 

  When Saving against one of these Toxins, the victim can make a Save every Minute \myital{before} they are affected by the damage.  If they make the Save, they do not take damage that Minute, BUT the Save does not necessarily end the effect of the Toxin.  In order to shake off the effect of the Toxin, the victim must make a number of  rolls based on the number in parenthesis (1 for Iron, 2 for Silver, and 3 for Gold).  Otherwise, the Toxin must run its course.

  When using a Toxin, you must make a \RS : \DEX or risk poisoning yourself.  Bravos do not need to make this roll.

  \example {
    Andre Preneur (Grit 7, Flesh 6) drinks a Silver Toxin in a cup of wine.  The Arbiter rolls the duration in secret (d8) and gets 7(!) - so the Toxin will course through his veins for 7 Minutes.  The Arbiter starts a timer. Andre rolls a Save and fails - the Arbiter rolls d8 and gets a 6, so Andre takes 6 points of damage, and he has 1 Grit and 6 Flesh left.  His friends run around trying to find a Leech but the clock is ticking.  Another Minute passes, Andre rolls another Save ... and makes it!  2 Minutes have passed (5 to go).  His compatriots decide to try to let the Toxin run its course.  The 3rd Minute passes, and Andre rolls his Save again ... and fails.  The Arbiter rolls 8 damage.  He is now Dying with 4 Minutes left of the Toxin.  He will need to roll his Death Die every time he takes damage from the Toxin for the next 4 Minutes, unless he can make 2 consecutive Saves
  }

  \newpage

  \mysubsection{Potent Waters (Acids)}{leech-chymistry-acids}

  In addition to Toxins, you can create various kinds of acids.  Acids are always treated as if they were unguents (though they are on the liquid side), and will dissolve flesh, stone, wood, or metal.  They are often used for etching, ruining locks, and pouring on people you hate.  Like Toxins, you must make a \RS : \DEX when using acids or risk pouring them on yourself.  Bravos do not need to make this roll.

  The efficacy of the acid depends on the type of coin you spend:

  \mytable{X X} {
    \thead{\COST} & \thead{Effects} \\
  } {
    500\FE  & 1 \\
    500\AG  & 2 \\
    500\AU & 3 \\
  }

  You can pick one of the effects only once (so if you spend Gold, you can get all 3 effects)
  \mybullet {
    \item if it hits Flesh, deals d4 damage for d4 Minutes. If it hits Armor, removes 1 \UD every Minute. 
    \item melt an area 100cm cubed of wood, metal, or stone
    \item create an acrid plume of smoke that causes coughing and choking for Minutes to everything Nearby (-2 to \RO and \RB attempts)
  }

  You can sunder your Shield to negate the effect of acids thrown on you.  


  \mysubsection{Tonics}{leech-chymistry-tonics}

  \CHYMISTRY[
    Name=Chyme's Nerve Tonic,
    Link=chymistry-chymes-nerve-tonic,
    Cost=500\FE,
    Duration=until Bivouac,
    Toxin=No,
    Narcotic=\MAX 1
  ]

  You make all of your \RO and \RB checks at +4, but you can never take a Breather - you're far too restless.  

\CHYMISTRY[
  Name=Cuckhold's Courage,
  Link=chymistry-cuckhold-courage,
  Cost=500\FE,
  Duration=0,
  Toxin=No,
  Narcotic=\MAX 3 
]

Drinking Cuckhold's Courage restores Grit when imbibed during a Breather.  For every bottle of Cuckhold's Courage drunk, the drinker's Grit is healed or increased (even above Max Grit!) by d6.  It cannot heal Flesh (only Grit).  Narcotic



\CHYMISTRY[
  Name=Fulcanelli's Clarifying Elixir,
  Link=chymistry-fulcanelli-clarifying-elixir,
  Cost=500\AG,
  Duration=until Bivouac/0,
  Toxin=No,
  Narcotic=No 
]


Renders you almost immune to any spells or effects from the Mind paradigm.  If you wouldn't normally get a Save against the effect, you now do.  If you *do* get a Save against the effect, the Save is a 6 in 6 (roll 2d6, and you only fail on a snake eyes).  If taken while under the effect of a Mind spell, immediately gives the imbiber a Save as above and ends its Duration.  Can be used to break the effects of the Philter of von Fuchs (it is rumored that Fulcanelli was under the sway of von Fuchs)

\CHYMISTRY[
  Name=Liebnitz Purgation,
  Link=chymistry-liebnitz-purgation,
  Cost=500\AG,
  Duration=0 ,
  Toxin=No,
  Narcotic=No 
]

If imbibed while under the effects of an ingested Toxin (Tonic, drunk Powder, or Brew) the poison will be immediately vomited forth intact and will cease to affect the victim.  

\CHYMISTRY[
  Name=Philter of von Fuchs,
  Link=chymistry-philter-von-Fuchs,
  Cost=varies,
  Duration=varies ,
  Toxin=Yes,
  Narcotic=No 
]

When imbibed, the drinker will fall under the sway of whoever gave them the tonic.  Treat as a \mylink{Charm}{wizardry-charm} spell.  Save Negates.  The duration depends on the type of coins spent:  Iron, Days; Silver, Weeks; Gold, Months.  Can be broken by spells, rituals, etc. that relieve effects of the Mind.

  \mysubsection{Powders}{leech-chymistry-powders}

  \CHYMISTRY[
    Name=Dastin's Basic Talc,
    Link=chymistry-dastins-basic talc,
    Cost=500\FE,
    Duration=0 ,
    Toxin=No,
    Narcotic=No 
  ]


  Sprinkling this powder on something covered in acid immediately neutralizes the acid (damage, etc).


  \CHYMISTRY[
    Name=Mermaid's Kiss,
    Link=chymistry-mermaids-kiss,
    Cost=500\AG,
    Duration=0 ,
    Toxin=Yes,
    Narcotic=No 
  ]


  When used, the recipient stops breathing for d6 Hours.  They can feign death or travel underwater, are not affected by inhaled powders or gases, and are unable to speak or cast spells.  Unwilling victims get a Save (as if this were a Toxin)



  \CHYMISTRY[
    Name=Powdered Bezoar,
    Link=chymistry-powdered-bezoar,
    Cost=500\FE,
    Duration=0 ,
    Toxin=No,
    Narcotic=No 
  ]


  When sprinkled on a food or into a beverage, has a 4-in-6 chance of neutralizing any Toxin contained inside.  The roll is made in secret by the Arbiter.


  \CHYMISTRY[
    Name=Woundseal,
    Link=chymistry-woundseal,
    Cost=500\FE,
    Duration=0 ,
    Toxin=No,
    Narcotic=No 
  ]


  Sprinkling this powder on a wound stops all effects of Bleeding, like the Leechcraft skill \mylink{Staunch}{leechcraft-staunch}


  \mysubsection{Ungeants}{leech-chymistry-ungeants}
  \CHYMISTRY[
    Name=Boyle's Sharpening Paste,
    Link=chymistry-boyles-sharpening-paste,
    Cost=500\FE,
    Duration=0 ,
    Toxin=No,
    Narcotic=No 
  ]
  When rubbed on the blade of a stabbing or cutting weapon, the weapon deals +d12 the next time damage is rolled.  The oil rubs off after the strike.


  \CHYMISTRY[
    Name=Brahe's Efficacious Sealant,
    Link=chymistry-brahes-efficacious-sealant,
    Cost=500\AU,
    Duration=0 ,
    Toxin=No,
    Narcotic=No 
  ]
  A strong, fast-drying paste. Capable of bonding stone, glass, wood, or metal (but not flesh). Lasts potentially forever.  Can cover an area roughly 10cm square.


  \CHYMISTRY[
    Name=Faivre's Aqua Grease,
    Link=chymistry-faivres-aqua grease,
    Cost=500\FE,
    Duration=0 ,
    Toxin=No,
    Narcotic=No 
  ]
  A pale grease that can be rubbed over any equipment to completely protect it against damage from water exposure

  \CHYMISTRY[
    Name=Tesla's Silver Wash,
    Link=chymistry-teslas-silver wash,
    Cost=500\AG,
    Duration=0 ,
    Toxin=No,
    Narcotic=No 
  ]
  When applied to a blade no larger than a longsword, the weapon becomes imbued with silver permanently.  Requires an ingot of silver in addition to the normal cost of 500\AG.

  \newpage

  \CHYMISTRY[
    Name=Wei Boyang's Alkahest,
    Link=chymistry-wei boyangs-alkahest,
    Cost=500\AU,
    Duration=0 ,
    Toxin=No,
    Narcotic=No 
  ]

  This oil will dissolve any adhesive (including Brahe's Efficacious Sealant).  Can cover an area roughly 10cm square.


\mysubsection{Sera}{leech-chymistry-sera}

  \CHYMISTRY[
    Name=Al-Farabi's Calming Injection,
    Link=chymistry-al-farabis-calming-injection,
    Cost=500\AU,
    Duration=0 ,
    Toxin=Yes,
    Narcotic=No 
  ]

  When injected, the creature is becalmed. The injected creature ceases immediately to be Enraged, Shaken, Frenzied, or Disgusted.  If the creature is Zoological, they become passive and docile.  Creatures under the effect of the Calming Injection cannot attack unless they are attacked first.  Lasts for Hours. Unwilling creatures get a Save to negate.

  \CHYMISTRY[
    Name=Davy's Soothing Anesthetic,
    Link=chymistry-davys-soothing-anesthetic,
    Cost=500\AG,
    Duration=0 ,
    Toxin=Yes,
    Narcotic=No 
  ]

  When injected, the creature feels any pain as pleasure for Hours.  Often surreptitiously given to those undergoing torture, or going under the knife for surgery.  Attacks against the recipient ignore Grit as the mind loses the cues to shift away from painful events.  


  \CHYMISTRY[
    Name=Grimm's Stupurous Preparation,
    Link=chymistry-grimms-stupurous-preparation,
    Cost=500\AU,
    Duration=0 ,
    Toxin=Yes,
    Narcotic=No 
  ]


  When injected into a person or creature, the creature immediately falls into a slumber in all ways like \mylink{Sleep}{wizardry-sleep} (can't be awakened except by a slap, doesn't effect a creature of greater than 4 \HD) unless they Save

  If, however, the sera is injected into a solid food of some sort (like an apple), and the food is ingested, it's true power becomes manifest.  The creature falls into a deep slumber and cannot be awakened except by a Witch feeding the curse to a Hekaphage (see Occultism). While asleep the creature is in a state of suspended animation - they do not age, and do not need to eat or drink - but they can still be killed in the normal means (dagger through the heart, etc).  Putting a creature into suspended animation will stop the effects of progressing disease and toxins.  Unwilling creatures get a Save. 

  \CHYMISTRY[
    Name=Wordwarp,
    Link=chymistry-wordwarp,
    Cost=500\AU,
    Duration=0 ,
    Toxin=Yes,
    Narcotic=No 
  ]

  An oil that causes a form of dyslexia.  If the victim fails a Save, they are unable to read written words for Hours.  Often used on \mylink{Sorcerers}{sorcerers}.




} % end