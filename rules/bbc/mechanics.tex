% (c) 2020 Stefan Antonowicz
% Based off of tex found at https://github.com/ludus-leonis/nipajin
% This file is released under Creative Commons Attribution-NonCommercial-ShareAlike 4.0 International License.
% Please do not apply other licenses one-way.

\renewcommand{\yggMechanics}{%
  \mychapter{Core Mechanics}{core-mechanics}
}

\renewcommand{\yggMechanicsText}{%

  \end{multicols}

  \mybold{Important: this book is supplemental to the Core Rules - read that book first!}

  \mysection{Abbreviations}{abbreviations}

  \example{
    \mylist {
      \item \hrulefill
      \item \DICE\bgspace The number of dice used to cast the spell 
      \item \hrulefill
      \item \SUMDICE\bgspace The sum of the dice used in casting the spell
      \item \hrulefill
      \item \MOD\bgspace A modifier to the roll you make. Can be positive or negative.
      \item \hrulefill
      \item \DURATION\bgspace The Duration of the spell - a length of time (Combat, \SUM Minutes, Session, etc); Concentration; or Markovian)
      \item \hrulefill
      \item \LENGTH\bgspace The length of time it takes you to perform a spell or ability.  Default is 1 Maneuver
      \item \hrulefill
      \item \COST\bgspace The number and type of coins worth of materials required for the spell, ritual, etc. If the type of coin isn't specified (iron, silver, or gold) it will be the coin appropriate for the Civilization (small, medium, or large)
      \item \hrulefill
      \item \SUCCESS\bgspace The number of successes you need for the ritual, miracle, etc. to take effect.  The number of "good" rolls must be greater than or equal to the number of "bad" rolls i.e. if you're rolling a \KNACK, a "good" roll is when you roll less than or equal to x-in-6; if you're rolling a \POOL or \UD, a "good" roll is anything but a 1 or 2.
      \item \hrulefill
      \item \TARGET\bgspace Who or what you can target, and how far away
      \item \hrulefill
      \item \COUNTER\bgspace If the spell can be countered by another spell (and if so, what spell counters it).
      \item \hrulefill
      \item \PARADIGM\bgspace  The Paradigm of the spell - what kinds of spell it is
      \item \hrulefill
      \item \KEYWORD\bgspace Any Keyword(s) associated with the spell
      \item \hrulefill
      \item \SAVE\bgspace Whether or not a victim gets a Save vs. Hexes (Yes or No).  The result of a successful Save will be in the spell's description

    }
  }

  \newpage 

  \begin{multicols}{2}

  \mysection{Duration}{duration}

  \mysubsection{Markovian}{duration-markovian}
  
  Markovian spells take effect immediately, but have a random duration depending on the number of \DICE invested in the casting:
  \mytable{c c} {
    \thead{\DICE} & {Duration} \\
  } {
    1 & d4 \\
    2 & d6 \\
    3 & d8 \\
    4 & d10 \\
    5 & d12 \\
    6+ & d16 \\
  }

  At the bottom of a Moment, the victim must \RS with the die associated with the duration.  If they fail (roll a 1 or a 2), the effect immediately ends.  If you don't fail, the die moves \DCDOWN.

  \mysubsection{Concentration}{duration-concentration}
  
  Concentration spells require the caster's continued attention to maintain.  Concentration is broken (and the spell ends) if the target moves out of range or line of sight, or if the caster is distracted at the Arbiter's discretion (by taking damage, being hit by a spell, knocked over, etc.).  The caster can walk slowly and still Concentrate, but cannot run.  They could whisper but not speak or yell.  Obviously, the caster can voluntarily end a Concentration spell simply by ceasing to concentrate on it. 

  \mysubsection{Other}{duration-other}

  \mylist {
    \item A duration of "Combat" indicates a spell or effect that will end at the end of Combat, before you take a Breather.
    \item A duration of "\SUM Minutes" indicates a spell or effect that will last a number of real world minutes as shown on the di(c)e.  The Arbiter should use a stopwatch or clock to measure the passage of time.  The Sorcerer, Leech, Witch, or Mystic can ask how many minutes are left at any time and get an answer, but the Arbiter is under no obligation to volunteer how much time remains if she doesn't want to.
    \item A duration of "Session" means the spell lasts for the entire Session.  It will need to be recast at the start of the next Session if you want the effect to continue.
  }

  \mysection{Keywords}{keywords}

  \mysubsection{Contested}{keyword-contested}

  Contested spells are a \RB between the spell caster and a Monster.  The caster's \RB is their Primary Stat ( \INT for Magicians and \FOC for Devotees - don't forget to add your \LVL to your roll!).  The target's \RB will be defined in the spell.  
  \example{
    The Sorcerer spell \mylink{Battering Beam}{wizardry-battering-beam} is a \RB between the spell caster and the target's \VIG with a -\DICE penalty.  The Sorcerer will \RB : \INT + \LVL and the victim will need to \RB : \VIG - \DICE, plus any other modifiers. 
  }

  \mysubsection{Splittable}{keyword-splittable}

  Spells that can be split can have their dice split up among up to \DICE targets.  Each target resolves the effect of the dice on them separately, but the \DICE are pooled when considering Mishaps, Calamities, and Ruin.

  \example {
    The Sorcerer spell \mylink{Charm}{wizardry-charm} is Splittable and allows you to "...[e]nsorcel one or more Monsters whose \HD are less than or equal to \DICE". If you had a pool of 4 Blood Dice, you could use 3 \DICE on 1 Monster and 2 on another, 1 [die] on 4 Monsters, etc. with the additional requirement that the number of \DICE used on the Monster is equal to or greater than their \HD
  }

  \mysubsection{Hammerspace}{keyword-hammerspace}
  \flavor {
    "... a fan-envisioned extradimensional, instantly accessible storage area in fiction, which is used to explain how animated, comic, and game characters can produce objects out of thin air." 
  }

  Spells that make use of hammerspace allow you to place objects in an "out of game" area until they can be retrieved.  Hammerspace objects can't be stolen, used, or obtained by others - but certain spells and rituals can tell unsavories whether or not you have things stored in Hammerspace.  You can't put one Hammerspace object in another (they repel like magnets). Objects in Hammerspace immediately return to reality when you die.

  \mysubsection{Purge}{keyword-purge}
  
  When something is "purged", it removes permanent effects in addition to temporary effects.  For example, the Leechcraft of \mylink{Bonesetting}{leechcraft-bonesetting} purges a non-serious Physical wound.  

  \newpage

  \mysection{Mortals}{mortals}

  Mortals are marked by the sign of Kib, the \myital{noumenon} - the soul or spirit, the impenetrable "thing-in-itself", which exists despite their inability to perceive or sense it.  The existence of this soul means that humankind are Hallowed, sacred to \TheAuthority. Hallowed things stand in opposition to Magick, a phenomenon whose rules are bound in Chaos rather than Order.  Only through the manipulation of the \mybold{Four Cruces} are the Hallowed permitted to wield the Magick denied to them by the Sign of Kib.

  \mysubsection{The Four Cruces}{mortals-four-cruces}

  \myhighlight{Blood}{mortal-crux-blood}

  The Crux of Blood is the Mortal power to control Magic in its raw form, the way the hand shapes the clay to make the vessel. Blood is the domain of \mylink{Sorcerers}{sorcerers}

  \myhighlight{Faith}{mortal-crux-faith}

  The Crux of Faith is the Mortal power to climb the Tree of Ygg and sit feet of the Authority, as a kite is borne by the wind to reach the clouds. Faith is the domain of \mylink{Mystics}{mystics}.

  \myhighlight{Knowledge}{mortal-crux-knowledge}

  The Crux of Knowledge is the Mortal desire to eat the fruits of the Tree of Ygg to gain true understanding over the effects of Magic, as the flame consumes fuel to produce heat. Knowledge is the domain of \mylink{Leeches}{leeches}.

  \myhighlight{Mojo}{mortal-crux-mojo}

  The Crux of Mojo yields to Magick in a lover's embrace, so it might both wield Magic's power and simultaneously be wielded by it.  Mojo is the domain of \mylink{Witches}{witches}.

  \cbreak

  \mysubsection{The Eight Paradigms}{mortals-eight-paradigms}

  Magic is divided into 8 Paradigms

  \mylist {
    \item \myanchor{\mybold{Biomancy}}{paradigm-biomancy}  Magic that affects or utilizes biological components

    \item \myanchor{\mybold{Death}}{paradigm-death} Magic that affects and interacts with the dead (Necromancy)
    
    \item \myanchor{\mybold{Elements}}{paradigm-elements} The basic 4 (air, fire, earth, water) as well as other "elemental" forces (acid, lightning, etc.)
    
    \item \myanchor{\mybold{Entropy}}{paradigm-entropy} Chaos and disorder; making things more chaotic or removing chaos from a system
    
    \item \myanchor{\mybold{Force}}{paradigm-force} Raw magical power that affects the material world in some way
    
    \item \myanchor{\mybold{Grace}}{paradigm-grace} The power of the Authority
    
    \item \myanchor{\mybold{Mind}}{paradigm-magic} Magic that affects the mind, including illusions and enchantments
    
    \item \myanchor{\mybold{Prophesy}}{paradigm-prophesy} Predicting the future, divining the present, unearthing the past
  }

  \mysubsection{Mortal Magick}{mortal-magick}

  \myhighlight{Sorcerers}{mortal-magic-sorcerers}

  \mybullet {
    \item \mylink{Wizardry}{sorcerer-wizardry} demands Blood. Old school, fireball-in-the-hallway, lightning bolt at level 5 spell casting.  
    \item \mylink{Inscription}{sorcerer-inscription} requires \mylink{Research Dice}{sorcerer-research}.  Can only be performed during Sabbaticals.  Scribing fetishes and grimores; researching True Names; writing sigls.
    \item \mylink{Staff Magic}{sorcerer-staff-magic} requires \mylink{Research Dice}{sorcerer-research}. Create a wizard's staff during a Sabbatical. 
  }

  \myhighlight{Leeches}{mortal-magic-leeches}
  \mybullet {
    \item \mylink{Leechcraft}{leech-leechcraft} requires Knowledge. Setting bones, patching Flesh, and delaying the onset of toxins. 
    \item \mylink{Chymistry}{leech-chymistry} requires \mylink{Research Dice}{leech-research}. Create potions, acids, and toxins during a Sabbatical.
    \item \mylink{Medicinals}{leech-medicinals} require \mylink{Research Dice}{leech-research}.  Cure diseases and addiction during a Sabbatical.
  }  

  \myhighlight{Witches}{mortal-magic-witches}
   \mybullet {
    \item \mylink{Charms}{witch-charms} use Mojo.  Hedge magic; hexes; and cantrips
    \item \mylink{Necromancy}{witch-necromancy} uses Mojo.  Speaking with the dead; healing Flesh (for a price); manipulating corpses. 
    \item \mylink{Occultism}{witch-occultism} requires spending Cunning Dice during a Sabbatical.  Bind familiars, curse enemies, and even bring the dead back to life.
  } 

  \myhighlight{Mystics}{mortal-magic-mystics}
  \mybullet {
    \item \mylink{The Seven Sacraments}{mystic-seven-sacraments} are bestowed by the Grace of the Authority. 
    \item \mylink{Liturgies and Invocations}{mystic-liturgies-invocations} demand Faith in the Small Gods.
    \item \mylink{Miracle Working}{mystic-miracle-working} requires Faith Dice during a Sabbatical. Create holy water and relics; call Crusades; and build Golems.
  }  

  \cbreak
  
  \mysection{Fae}{fae}

  The Fae races are \myital{phenomenal} - they lack a definable essence, since they are born of Magic and Chaos (though they stand outside of it in the same way a fish swimming in a stream is both a part of and separate from the water). The Fae exist so long as Magic exists. They are unmarked by the Authority (like familiars, constructs, and the undead), and thus stand in opposition to Mortals. For this reason they are considered Unhallowed.  By the Grace of the Authority, Mortal Mystics can banish the Unhallowed from their midst.


  \mysubsection{Fae Magick}{fae-magick}

  \myhighlight{Spriggans}{fae-magic-spriggans}
  \mybullet {
    \item \mylink{The Forgotten}{spriggan-forgotten} (the \mybold{Obliterated} and the \mybold{Abandoned}) can be made manifest using Remembrance Dice.
    \item \mylink{Sword Magic}{spriggan-sword-magic} can be created and imbued with long dead magics with the help of the Forgotten, Witches, Sorcerers, and Mystics.
  } 

} %end