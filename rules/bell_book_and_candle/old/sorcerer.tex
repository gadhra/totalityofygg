% (c) 2020 Stefan Antonowicz
% Based off of tex found at https://github.com/ludus-leonis/nipajin
% This file is released under Creative Commons
% Attribution-NonCommercial-ShareAlike 4.0 International License.
% Please do not apply other licenses one-way.

\renewcommand{\yggSorcerer}{%
  \mychapter{Sorcerers}{sorcerers}
}

\renewcommand{\yggSorcererText}{%

\mysection{Blood}{sorcerer-blood}

  \mytable{l X}{
    \thead{Level} & \thead{Blood} \\
  }{
    1 & 1d6 \POOL \\
    2 & 2d6 \POOL \\
    3 & 3d6 \POOL \\
    4 & 4d6 \POOL \\
    5 & 5d6 \POOL \\
    6 & 6d6 \POOL \\
    7 & 7d6 \POOL \\
    8 & 8d6 \POOL \\
    9 & 9d6 \POOL \    
  }



Sorcerers wield Wizardry through Blood, represented by a Blood Die - a single d6.  You have \LVL number of Blood Die.

When you cast a spell, you can roll any number of Blood Dice.  All the dice have to be rolled at once when you cast the spell:

\mybullet {
  \item If you roll any triples, roll on the \mylink{Mishap
table}{table-mishap}. The spell works.
  \item If you roll any quadruples, roll on the \mylink{Calamity
table}{table-calamities}. The spell fails.
  \item If you roll any quintuples, roll on the \mylink{Ruin
table}{table-ruin}. The spell fails.  This will most likely kill you.
}

The Blood Die is a \POOL.  If you roll a 1 or a 2 on a Blood Die, you lose the die. You can restore 1 Blood Die when you Bivouac, or all your Blood Die with a longer rest - but you can also attempt to ride the Torrent.


\mysection{The Torrent}{sorcerer-torrent}

If you have 0 Blood Die in your pool, you can still attempt to perform
magic, though doing so is dangerous:

\example {
  \mybold{Torret}
  \RO : \INT plus d6 plus Modifiers

  ~\\
  
  \myital{Don't forget to add your \LVL, since this roll uses your \INT!}
}

If you succeed, you can add a Blood Die to the spell.  You can do this as
many times as you like for each spell. If you fail any \RO attempt, the
spell fails and you suffer a \mylink{Calamity}{table-calamities}.



\mysection{Components}{sorcerer-components}

You can harvest and use components to cast spells:

\example {
  \mybold{Harvest Component}
  \RO : \INT plus d6 plus d20 plus Modifiers

  ~\\
  
  \myital{Don't forget to add your \LVL, since this roll uses your \INT!}
}


If you cast a spell with a component, then you only burn a Blood die if you
roll a 1 (instead of a 1 or 2).  Components are Insignificant items, but
using a component consumes it in a puff of smoke / blue flame / swarm of
gnats etc.

You can say something is a component for a spell you choose.  For example,
you could declare that kobold's eyes are a component for the spell Sleep. 
The Arbiter will ask you to make a successful Lore check - if you make it,
then kobold's eyes are always a component for Sleep (note it on your
character sheet) for YOU and not for anyone else (someone else's version of
Sleep has a different way of casting, needs boggart boogers or whatever). 
Components are \UD 

Couple of rules

\mybullet {
  \item Components can only be harvested during a Breather, and have to be
from one of the Monsters you've (presumably) killed.  It has to be something
you can reasonably carry (Arbiter's choice) - basilisk tongue, ok.  Green
slime, not ok.
  \item Monsters only provide one component.  You can't say kobold's eyes
help Magic Missile and kobold's tongues help cast Battering Beam.
  \item You can only try once per Monster type per Breather, so if you kill
9 kobolds and 14 trolls then you get two shots - once on kobolds and once on
trolls.
  \item You can only harvest d4 \UD from a Monster no matter how many there
are - 9 kobolds or 1, you get d4 \UD
}


\mysection{Learning Spells}{sorcerers-learning-spells}

Your starting spells depend on your \INT:

  \mytable{X l l}{
    \thead{\INT} & \thead{Random} & \thead{Pick} \\
  }{
    d4 & 2 & 1 \\
    d6 & 4 & 2 \\
    d8 & 6 & 3 \\
    d10 & 8 & 4 \\
  }


You can either roll randomly (and get more spells) or pick from the list of
\mylink{Wizardry}{sorcerer-wizardry}.  These are all the spells you
understand.

One of these spells (your choice) is etched on the inside of your cranium,
"...like moth larvae burrowing through the bark of a tree."  This spell can
be cast without speaking or moving.  If someone cuts off your head and
scoops out your brains, they can read (and learn) the spell too.

The rest of your spells are inscribed in a
\mylink{Grimoire}{sorcerer-grimoires}.  You can copy these spells to
\mylink{Fetishes}{sorcerer-fetishes} using
\mylink{Inscription}{sorcerer-inscription}.

If you have a spell in your skull or on a fetish in your possession, you can
cast the spell without digging through your grimoire (meaning you can cast a
spell in a single Maneuver). 

Searching through a grimoire in your hands adds 1 Maneuver to the spell
before you can cast it (so it would take 2 Maneuvers to find and cast a
spell).  If the grimoire is in your backpack, you would need to dig it out
(1 Maneuver), find the spell (1 Maneuver), and cast it (1 Maneuver).  

Every time you gain a \LVL, you can etch an additional spell inside your
cranium. This must be a spell in your possession (in a grimoire or on a
fetish)

You \mybold{do not} learn a new spell each time you gain a \LVL.  You must
find or discover new ones.


\mysection{Research}{sorcerer-research}

Research is used to perform \mylink{Inscription}{sorcerer-inscription}, and
can be used to help Leeches practice \mylink{Chymistry}{leech-chymistry}. 
Your Research is equal to your maximum \INT (so if you have a d12 \INT, you
have 12 Research). You may only perform Research in a Civilization with a
library.

If you take a Sabbatical, the number of Research Die you have is tripled (so
12 Research becomes 36).

Use your Research to "build" one or more \KNACK called Research Die. You can
"build" as many Research Die as you choose up to your Research total. For
example, if you have 12 Research, you could create:

\mybullet {
  \item 2 6-in-6 Research Die
  \item 3 4-in-6 Research Die
  \item 6 2-in-6 Research Die
  \item etc.
}

Remember that a Research Die is a \KNACK.  If your Research Die is a 2-in-6,
you get a success if you roll a 1 or a 2 on a d6; if the die is 6-in-6, you
roll 2d6 and only fail on a 2 (snake eyes).  You can only roll each Research
Die once per Sabbatical.

Research Dice can be traded to \mylink{Sorcerers}{sorcerers} as well as to
other Leeches (knowledge is shared, after all). You can give as many of your
Research Dice to different people as you wish, but you can only accept a
number of Research Dice up to your \LVL-1.

\mysection{Wizards' Duel}{sorcerer-wizards-duel}

Certain spells can be countered by other spells - for example,
\mylink{Balthazar's Breathtaking
Blast}{wizardry-balthazars-breathtaking-blast} can be countered by
\mylink{Mighty Lungs}{wizardry-mighty-lungs}, and
\,mylink{Invisibility}{wizardry-invisiblity} can be dispelled by the wisps
summoned in the \mylink{Fool's Fire}{wizardry-fools-fire} spell. 

If you attempt to counter a spell and the Sorcerer who cast it is not
present (that is, not somewhere Close, Nearby, Far-Away, or Distant) you
automatically succeed.  Otherwise, you enter into a duel with the other
Sorcerer.

Each Sorcerer must \RB : \INT, adding their \LVL and the \DICE invested in
the spell as well as any other modifiers.  

\example {
  \mybold{A Wizards' Duel is}
  \RB : \INT plus \LVL plus \DICE
}


The winner's spell stays, the loser's spell goes.  Put another way: 

\mybullet{
  \item If you are countering a spell and you win, the spell you're
attempting to counter is dispelled and your spell takes effect
  \item If you are countering a spell and you lose, the spell you're
attempting counter remains and your spell fizzles with no effect.
}

If you attempt to counter a spell and you roll a
\mylink{Calamity}{sorcerer-blood} or \mylink{Ruin}{table-ruin}, the
counterspell doesn't work. 


\mysection{Grimoires}{sorcerer-grimoires}

Grimoires are solid, with thick vellum pages and a sturdy cover. Special
runes and symbols trap spells inside cages of crystallized thought. Each
book contains enought room for 10 spells. Some spells must be stored across
several pages for safety, so the books contain more than 10 pages, and have
plenty of room for notes, ledgers, or sketches - and curses, hexes, coded
and cryptic entries written with poisonous and hallucinogenic inks.
Grimoires start in a waterproof, acid- and fire-resistant bag. Outside the
bag, they are not waterproof and are quite flammable. See the section on
\mylink{Inscription}{sorcerer-inscription} for more details.

\cbreak

\mysection{Fetishes}{sorcerer-fetishes}

Fetishes are inscribed with a single spell from your grimoire, cranium, or
another fetish.  All fetishes have a \UD of d4 - when the \UD is exhausted,
the magical words disappear from the fetish.  A Sorcerer's skull counts as a
fetish, though obviously the brain can't be in the way if you want to read
it.  See the section on \mylink{Inscription}{sorcerer-inscription} for more
details.

You may cast the spell from the Fetish with any number of Blood you choose,
OR you may forgo the Blood die and roll a single d6 for the effect.  

\newpage

\mysection{Wizardry}{sorcerer-wizardry}



\newpage


\newpage




} %end
