% (c) 2020 Stefan Antonowicz
% Based off of tex found at https://github.com/ludus-leonis/nipajin
% This file is released under Creative Commons Attribution-NonCommercial-ShareAlike 4.0 International License.
% Please do not apply other licenses one-way.

\renewcommand{\yggSkillsSaves}{%
  \mychapter{Skills and Saves}{skills-and-saves}
}

\renewcommand{\yggSkillsSavesText}{%
  \mysection{Skills}{skills}
  
  Skills are a \KNACK. There are seven basic skills.  Unless noted in your Flavor or Species, all of your skills start at 1-in-6 (you need to roll a 1 on a d6 to use it successfully).  When you create your character you can choose to increase one of these Skills by 1 based on your character's history ("my Pooka is a lucky charm on the ship \myital{The Dowager}, so he should start with a +1 to Salt") or give yourself a 1 in a new Skill ("my Soldier apprenticed with a blacksmith so I should get Blacksmith as a skill").  Final approval is up to the Arbiter.  The Skill has to be a 1 in order to improve it at the start of the game (i.e. if you played a Soldier with a starting Travel of 2, you can't raise it to 3). 

  \mylist {
    \item \myanchor{\mybold{Bushcraft}}{skill-bushcraft} tracking, hunting, camping, finding water, survival, making fires, ranger shit
    \item \myanchor{\mybold{Eyeball}}{skill-eyeball} estimate something's value or weight; size somebody up; judge distance, quality, etc 
    \item \myanchor{\mybold{Listen}}{skill-listen}  listen at doors, remember something the Arbiter told you that you've forgotten, tell if someone's lying
    \item \myanchor{\mybold{Lore}}{skill-lore}  general ancient knowledge - esoteric religions, what the weird writing says, history of this magic item, whether or not something is a component 
    \item \myanchor{\mybold{Math}}{skill-math} architecture, sloping rooms, construction, where a secret door might be 
    \item \myanchor{\mybold{Salt}}{skill-salt} swimming, sailing boats, having sea-legs, fishing, navigating by the stars, tying knots
    \item \myanchor{\mybold{Travel}}{skill-travel} estimating food and water for a convoy; riding horses and camels; mountain climbing 
  }

  These are only the basics skills that everyone gets.  As stated above, you can add another skill to your character sheet based on your character history (you were a Haberdasher in Lankhmar and now you have Skill: Tailoring).  You may also find additional skills by reading books, apprenticeships, etc (for example, you might find a book on teacups or something and you spend a Sojourn reading it and now you have the Skill: Tea Ceremony).
  
  \cbreak

  \mysubsection{Advancing Your Skills}{skills-advance}
  Every time you use a skill \mybold{and succeed}, put a check mark next to the skill. When you take a longer rest (Sojourn or Sabbatical), you can choose to try to level your skill.  Roll a number of d6 equal to your current Skill (so if your skill is 3-in-6, roll 3d6). If the result is equal to or less than the number of checks, your skill goes up a level. Erase all your checks and start again.

  \myital{Ex: Gary the Caravan Guard got lost in the desert and used Bushcraft five times and only 1 time did he get hopelessly lost and had a hand chewed off by a Great Orm. So that's 4 successes. He currently has a 2 in 6 skill level when he makes it back to Lankhmar.  He decides to try to level his skill, rolls 2d6, and gets a 3. Since 3 is less than 4, his skill goes up a level. Yay Gary!}

  \mysubsection{Skill Rules}{skills-rules}

  \mybold{1. When will the Arbiter call for a Skill check?}
  
  \mylist {
    \item When the difference between success and failure would be interesting.
    \item When it's a skill that's not interesting enough to describe but still shouldn't be an automatic success.
  }


  \mybold{2. Oh, she didn't succeed? I try too.} 

  The Skill roll doesn't just determine success, it determines if something is possible.  If you fail at something like lighting a fire in a dungeon (Bushcraft), it means that it's simply not possible in these conditions.  Might have to do with your actions, might not. It was too damp in the dungeon. You got the tinder wet.  The wood is sodden. Who knows, but it's just not possible, you tried, failed, and now you know. 

  If the sailor is going at the Salt roll, well, of course there's a better chance of success. Shouldn't they be able to attempt something a less skilled adventurer tried and failed at? Well, I guess that adventurer fucked up and should have waited for the professional to give it a shot, but they didn't and they tore the mainsail, crumbled the crevasse they were trying to jump over, scared away the deer, etc. This may be permanent, or for a set amount of time, common sense should prevail.

  \mybold{3. When are individual rolls allowed?} 

  The first roll still determines if it's possible. If that roll succeeds, the others follow her lead, although with less expertise. A bonus may apply. Rolls beyond the first don't cause it to become impossible unless it really makes sense (if adventurer 1 succeeds, it's possible, and adventurer 2 failing doesn't make it impossible thereafter, like adventurers 3 and 4 can't swim anymore)


  \mysubsection{Non-Standard Skills}{skills-non-standard}

  As stated earlier, Skills can pretty much be anything (Haberdashery, Tea Ceremony, Nose-picking, etc. etc).  Here are a few other Skills mentioned in these rules.

  \mylist {
    \item \myanchor{\mybold{Veins Lore:}}{skill-veins-lore}   Any time you encounter something in the \mylink{Veins}{murk-veins-of-the-earth} you might not understand, you can try this roll.  Some examples:  remembering what kind of gift is appropriate to give to an Ambassodile, what that smell is in the darkness, figuring out what kind of architect this Janeen is, etc.  If you are a Murk and were a slave of the culture you're rolling against, increase your roll by 1 (1-in-6 becomes 2-in-6, etc)
    
    \item \myanchor{\mybold{Tinkering:}}{skill-tinkering}   The skill of fixing, understanding and manipulating small devices and tools.  A shittier, non-magic version of Knave Whispers, most often  used by scavengers to disarm small mechanical traps or their triggers, picking simple locks, and to trying to repair broken stuff.  If you fail your Tinker roll the worst possible event happens i.e. the trap explodes for maximum damage, the weapon you're "fixing" is destroyed beyond repair, the improvised lockpick breaks and now no-one can pick the lock, etc.  You can use Tinker during a Bivouac to try to repair a \UD of Armor, but if you fail you make it 1 \UD \mybold{worse}

    \item \myanchor{\mybold{Linguistics:}}{skill-linguistics}   Any time you encounter a written language you can't read, you can try this roll.  If you succeed you get a general idea of what the text says - the Arbiter can give you a general idea of what the text says, but not specifics. You can't use this skill to read magical writing (scrolls, spell books, etc)

  }

  \mysection{Saves}{saves}
  You missed the poison needle in the chest, got too close to the dragon's flame, didn't kill the swamp hag before she pointed her magic wand at you and caused snakes to shoot out and try to bite you on your stupid face.

  You've got one chance left.  Every Class has three \KNACK called Saves that they can roll to avoid a horrible death:
  \mylist {
    \item \myanchor{\mybold{Hexes}}{save-hexes}  - this includes certain spells and magical effects. If a spell description says "Save for half damage", this is the Save you roll.
    \item \myanchor{\mybold{Toxins}}{save-toxins}  - various and sundry poisons and diseases.  If a poison or disease description says "Save or suffer d100 damage", this is the Save you roll.
    \item \myanchor{\mybold{Doom}}{save-doom}  - includes dragon's breath, the gaze of a medusa, or things that might kill you outright.  If a Monster's description says something like "Save or die", this is the Save you roll.
  }

  Saves are rolled on a d6.  Pick one Save as your "Edge" Save and one Save as your "Flawed" Save.  The remaining Save is an "Ordinary" Save.

  \mylist {
    \item Your Edge Save starts at 2-in-6.  Flawed and Ordinary Saves start at 1-in-6.
    \item Every level you can either improve your Edge OR your Ordinary Save by 1 (2-in-6, 3-in-6, etc ).  
    \item On odd levels after the first (Level 3, 5, 7, and 9) you can choose to improve your Flawed Save by 1 instead of your Edge or Ordinary Save (this doesn't include \mylink{Murks}{species-murk}, who can improve their Flawed Saves at any level)
  }

} %end