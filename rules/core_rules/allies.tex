% (c) 2020 Stefan Antonowicz
% Based off of tex found at https://github.com/ludus-leonis/nipajin
% This file is released under Creative Commons Attribution-NonCommercial-ShareAlike 4.0 International License.
% Please do not apply other licenses one-way.

\renewcommand{\yggAllies}{%
  \mychapter{Allies}{allies}
}

\renewcommand{\yggAlliesText}{%

  \mysection{Mortals}{allies-mortals}


  \flavor{And Kib grew weary of the second game, and raised his hand in the Middle of All, making the sign of Kib, and made Men: out of beasts he made them, and Earth was covered with Men.  \Tilde Lord Dunsany}


  Humans (mortals) are marked by the sign of Kib, the \myital{noumenon} - the soul or spirit, the "thing-in-itself", which exists despite their inability to perceive or sense it.  The existence of this vitality means that Humankind are Hallowed, possessing an undefinable quality that makes them Human. Hallowed things stand in opposition to Magick, a phenomenon whose rules are bound in Chaos rather than Order.

  Humans are pursued by Sish, the Destroyer of Hours, and fall beneath the sign of Mung, the God of All Deaths. In this way, Order is maintained.  Mortals cover every conceivable biome of Acheron - from the teeth of Vornheim to the lava pools of the Jungles of Klesh. 

  \mysubsection{Mortal Origins}{allies-mortal-origin}

  \myital{Note:  These are just suggestions - feel free to buck the rules!}

  \mybullet {
    \item  Roll a d6:  1-3 you're from the country, 4-5 a village, 6 the big city
    \item  Roll or pick from the list of \mylink{d30 Character Occupations}{appendixb-character-occupations}.  This is either what you were doing before, or what you were \myital{going} to do before you (presumably) ran away from home.
  }


  \cbreak


  \mysection{Fae}{allies-fae}


  \flavor{And within the dark circle in which the Freer stood making his curses were no unhallowed things, nor were there strangenesses such as come of night, nor whispers from unknown voices, nor sounds of any music blowing here from no haunts of men; but all was orderly and seemly there and no mysteries troubled the quiet except such as have been justly allowed to man.  And beyond that circle whence so much was beaten back by the bright vehemence of the good man's curses, the will-o'-the-wisps rioted, and many a strangeness that poured in that night from Elfland, and goblins held high holiday. For word was gone forth in Elfland that pleasant folk had now their dwelling in Erl; and many a thing of fable, many a monster of myth, had crept through that border of twilight and had come into Erl to see. And the light and false but friendly will-o'-the-wisps danced in the haunted air and made them welcome.  \Tilde Lord Dunsany}

  The Fae races are \myital{phenomenal} - they lack an undefinable essence, since they are born of Magick and Chaos (though they stand outside of it in the same way a fish swimming in a stream is both a part of and separate from the water). The Fae exist so long as Magick exists. They are unmarked by Kib (like familiars, constructs, and the undead), and thus stand in opposition to Mortals. For this reason they are considered Unhallowed.  By the Grace of the Authority, Mortal Devotees can banish the Unhallowed from their midst.

  \newpage 

  \mysubsection{Fae Origins}{allies-fae-origin}
  \myital{Note:  These are just suggestions - feel free to buck the rules!}

  \mybullet {
    \item If you are a \mylink{Pooka}{flavor-pooka}, you can hail from any city in Acheron.  Pooka are commonly sailors (they bring good luck), jesters, scouts, and tavern keepers - and rarely gamblers, since no one will play with them
    \item If you are a \mylink{Spriggan}{flavor-spriggan}, you originally hailed from Elfland - but can live in or near any town or city bordering the Great Wood.
    \item If you are a \mylink{Murk}{flavor-murk}, you originally hailed from the Veins of the Earth - but you can live in any town or city on the frontiers.
    \item If you are a \mylink{Night Child}{flavor-night-child}, you hail from any city in Acheron, or somewhere far away from smaller civilization.
  }

  \mysection{Character Creation}{allies-character-creation}
  
  \mybold{Step 1: Pick your \mylink{Tangible Stats}{tangible-stats} and \mylink{Intangible Stats}{intangible-stats}}
  
  For your Tangible Stats, take out a d10, a d8, a d6, and a d4.  Assign one of the dice to \VIG, \DEX, \INT, and \FOC, one die per stat ("bigger" dice are better)

  Each Intangible Stat starts at a d4 (though this may be modified by your  or Small God)

  \mybold{Step 2: Choose your Destiny}
  Decide if you're going to be \mylink{Mortal}{allies-mortal} or \mylink{Fae}{allies-fae}.  Roll or pick your Mortal or Fae origin.

  \mybold{Step 3: \mylink{Initial Skills}{skills}}

  Write down the 7 basic \mylink{Skills}{skills}: Bushcraft, Eyeball, Listen, Lore, Math, Salt, and Travel.  Each of these skills are a \KNACK that starts at 1-in-6.


  \mybold{Step 4: \mylink{Initial Saves}{saves}}

  Pick your Edge, Ordinary, and Flawed \mylink{Save Die}{saves}.  Each of these is a \KNACK.  Edge starts at 2-in-6, Flawed and Ordinary start at 1-in-6.

  \cbreak
  
  \mybold{Step 5: Choose your \mylink{Template}{template} and \mylink{Flavor}{flavor}}

  Pick a Template and a Flavor beneath that Template.  Adjust your stats as necessary.  Templates and Flavors  are described in the next section.


  \mybold{Step 6:  Choose your \mylink{Languages}{arbiters-resources-languages}}

  Everyone starts with Acheron, the "common" tongue. If you are Mortal, you also can speak one Dialect of your choice as you secondary language.  If you are Fae, your secondary language is:
  \mylist {
    \item \mybold{Pooka:}  Saltish or Birdsong
    \item \mybold{Murk:}  Silent Speech
    \item \mybold{Night Child:} Draconic
    \item \mybold{Spriggan:} Archaic, Fiendish, OR Seraphic
  }

  Finally,\RO using your \INT + d12.  If you roll a 20, gain one additional language; 21, two languages; and 23, three additional languages. 20: 1 additional; 21: 2 additional; 22: 3 additional.

  It's up to you if you can read AND write these languages.


  \mybold{Step 7:  \mylink{Flesh and Grit}{combat-flesh-grit}}

  Your starting Flesh is the \MAX value for your \HD as listed under your Template.  For example, Sellswords have a \HD of d10, so they start with 10 Flesh.  

  Roll your \VIG.  If the total is greater than your current Flesh, your Flesh is that number instead (i.e. if you roll a 6 and you have 4 Flesh, your Flesh is now 6). \mylink{Sellswords}{template-sellsword} (Soldiers and Barbarians) can add their \LVL to this roll, but no other modifiers.

  \mybold{Step 8: Additional Skill}

  Adventurers have been around the block a bit.  Increase one \mylink{Skill}{skills} by 1 (up to 2) or give yourself a 1 in a new Skill based on your character concept.

  \mybold{Step 9:  \myanchor{Starting Gear}{starting-gear}}

  In addition to any professional gear for your Flavor, you've got:
  \mylist {
    \item 2 iron pieces; 
    \item a backpack containing a worn bedroll;
    \item d4 \UD of personal provisions; and 
    \item a stained and patched waterproof cloak
  }

\newpage
}